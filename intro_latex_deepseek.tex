\documentclass{article}
\usepackage{amsmath} % Part of standard LaTeX for math
\usepackage{booktabs} % Part of standard LaTeX for tables

\title{A Beginner's Guide to Essential \LaTeX{} Elements}
\author{Your Name}
\date{\today}

\begin{document}

\maketitle

\section{Introduction}
This document introduces the basic elements of \LaTeX{} in a simple, beginner-friendly way. For each feature, we'll show both the code and its rendered output.

\section{Document Structure}
Every \LaTeX{} document has a basic structure:

\begin{verbatim}
\documentclass{article} % or report, book, etc.
\begin{document}
% Your content goes here
\end{document}
\end{verbatim}

The \texttt{documentclass} defines the type of document. Common classes are:
\begin{itemize}
    \item \texttt{article}: For short documents, papers
    \item \texttt{report}: For longer documents with chapters
    \item \texttt{book}: For books
\end{itemize}

\section{Sections and Subsections}
\LaTeX{} provides automatic section numbering. Here's how to create them:

\begin{verbatim}
\section{Main Section}
\subsection{Subsection}
\subsubsection{Subsubsection}
\end{verbatim}

This produces the section headings you see in this document.

\section{Text Formatting}
Basic text formatting commands:

\begin{verbatim}
\textbf{Bold text} \textit{Italic text} 
\underline{Underlined} \texttt{Typewriter}
\end{verbatim}

Rendered output: \textbf{Bold text} \textit{Italic text} \underline{Underlined} \texttt{Typewriter}

\section{Lists}
\LaTeX{} offers two main list types:

\subsection{Itemized Lists (Bullet Points)}
\begin{verbatim}
\begin{itemize}
    \item First item
    \item Second item
\end{itemize}
\end{verbatim}

Output:
\begin{itemize}
    \item First item
    \item Second item
\end{itemize}

\subsection{Enumerated Lists (Numbered)}
\begin{verbatim}
\begin{enumerate}
    \item First item
    \item Second item
\end{enumerate}
\end{verbatim}

Output:
\begin{enumerate}
    \item First item
    \item Second item
\end{enumerate}

\section{Mathematical Expressions}
\LaTeX{} excels at typesetting mathematics. There are two main environments:

\subsection{Inline Math}
Surround math with dollar signs:
\begin{verbatim}
The Pythagorean theorem: $a^2 + b^2 = c^2$
\end{verbatim}
Output: The Pythagorean theorem: $a^2 + b^2 = c^2$

\subsection{Display Math}
For centered equations on their own line:
\begin{verbatim}
\[ E = mc^2 \]
\end{verbatim}
Output:
\[ E = mc^2 \]

\subsection{Equation Environment}
For numbered equations:
\begin{verbatim}
\begin{equation}
f(x) = x^2 + 2x + 1
\end{equation}
\end{verbatim}
Output:
\begin{equation}
f(x) = x^2 + 2x + 1
\end{equation}

\section{Tables}
Creating tables in \LaTeX{}:

\begin{verbatim}
\begin{tabular}{|l|c|r|}
\hline
Left & Center & Right \\ \hline
A & B & C \\ \hline
1 & 2 & 3 \\ \hline
\end{tabular}
\end{verbatim}

Output:
\begin{tabular}{|l|c|r|}
\hline
Left & Center & Right \\ \hline
A & B & C \\ \hline
1 & 2 & 3 \\ \hline
\end{tabular}

For professional-looking tables, use \texttt{booktabs}:

\begin{verbatim}
\begin{tabular}{lcc}
\toprule
Item & Quantity & Price \\ \midrule
Apples & 10 & \$1.00 \\
Oranges & 5 & \$1.50 \\ \bottomrule
\end{tabular}
\end{verbatim}

Output:
\begin{tabular}{lcc}
\toprule
Item & Quantity & Price \\ \midrule
Apples & 10 & \$1.00 \\
Oranges & 5 & \$1.50 \\ \bottomrule
\end{tabular}

\section{Conclusion}
This document covered the basic elements of \LaTeX{}:
\begin{itemize}
    \item Document structure
    \item Sections and formatting
    \item Lists and tables
    \item Mathematical expressions
\end{itemize}

\section*{References}
For further learning:
\begin{itemize}
    \item The Not So Short Introduction to LaTeX2e: \\
    \texttt{https://tobi.oetiker.ch/lshort/lshort.pdf}
    \item LaTeX Wikibook: \\
    \texttt{https://en.wikibooks.org/wiki/LaTeX}
    \item Overleaf Documentation: \\
    \texttt{https://www.overleaf.com/learn}
\end{itemize}

\end{document}