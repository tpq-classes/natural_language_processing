\documentclass{article}
\usepackage{hyperref, amsmath, amsthm, algorithmic, graphicx, listings, minted, booktabs}
\usepackage[capitalize]{cleveref}

\newtheorem{theorem}{Theorem}
\newtheorem{lemma}{Lemma}
\newtheorem{proposition}{Proposition}

\title{Longstaff-Schwartz Least Squares Monte Carlo Method for American Option Pricing}
\author{Quant Tutorial Series}
\date{\today}

\begin{document}

\maketitle
\tableofcontents

\section{Theoretical Foundation}
\subsection{American Option Valuation Problem}
The value of an American option at time $t$ is given by the optimal stopping problem:
\begin{equation}
    V_t = \sup_{\tau \in \mathcal{T}_{t,T}} \mathbb{E}^\mathbb{Q}\left[e^{-r(\tau-t)} h_\tau(S_\tau) \mid \mathcal{F}_t\right]
\end{equation}
where $\mathcal{T}_{t,T}$ is the set of stopping times with values in $[t,T]$.

\subsection{Dynamic Programming Formulation}
The value function can be expressed recursively:

\begin{theorem}[Dynamic Programming Principle]
The American option value satisfies:
\begin{equation}
    V_t = \max\left\{h_t(S_t), \mathbb{E}^\mathbb{Q}\left[e^{-r\Delta t} V_{t+\Delta t} \mid \mathcal{F}_t\right]\right\}
\end{equation}
\end{theorem}

\begin{proof}
See Appendix \ref{app:dp_proof}.
\end{proof}

\subsection{Conditional Expectation Approximation}
Longstaff-Schwartz approximates the continuation value using least squares regression:

\begin{equation}
    \mathbb{E}^\mathbb{Q}\left[e^{-r\Delta t} V_{t+\Delta t} \mid S_t\right] \approx \sum_{j=1}^J \alpha_j \phi_j(S_t)
\end{equation}

where $\{\phi_j\}$ are basis functions.

\section{Implementation Guide}
\subsection{Algorithm Steps}
\begin{algorithmic}[1]
\STATE Generate $N$ paths of underlying asset $S_t$ under $\mathbb{Q}$
\STATE Initialize cashflow matrix $CF \gets h_T(S_T)$
\FOR{$t = T-\Delta t$ downto $0$}
\STATE Identify in-the-money paths $\mathcal{I}_t = \{i : h_t(S_t^i) > 0\}$
\STATE Regress discounted cashflows on basis functions:
$Y^i = e^{-r\Delta t} CF_{t+\Delta t}^i$, $X^i = \sum_j \alpha_j \phi_j(S_t^i)$
\STATE Estimate continuation value $\hat{C}_t(S_t^i) = \sum_j \hat{\alpha}_j \phi_j(S_t^i)$
\STATE Update cashflows: $CF_t^i = \begin{cases} 
h_t(S_t^i) & \text{if } h_t(S_t^i) > \hat{C}_t(S_t^i) \\
e^{-r\Delta t} CF_{t+\Delta t}^i & \text{otherwise}
\end{cases}$
\ENDFOR
\STATE Option value $\approx e^{-rT} \frac{1}{N} \sum_{i=1}^N CF_0^i$
\end{algorithmic}

\begin{figure}[h]
\centering
\includegraphics[width=0.8\textwidth]{lsm_flowchart}
\caption{LSM Algorithm Flowchart}
\label{fig:flowchart}
\end{figure}

\section{Code Implementation}
\subsection{Python Implementation}
\begin{minted}{python}
import numpy as np
from numpy.polynomial import legendre

# Parameters from Longstaff-Schwartz Table 1
S0 = 36    # Initial stock price
K = 40     # Strike price
r = 0.06   # Risk-free rate
sigma = 0.2 # Volatility
T = 1.0    # Time to maturity
N = 100000 # Number of paths
M = 50     # Time steps

def lsm_american_option(S0, K, T, r, sigma, N, M, degree=3):
    dt = T/M
    t = np.linspace(0, T, M+1)
    
    # Generate paths
    S = np.exp((r - 0.5*sigma**2)*dt + sigma*np.sqrt(dt)*np.random.normal(size=(N,M)))
    S = np.hstack([np.ones((N,1))*S0, S])
    S = np.cumprod(S, axis=1)
    
    # Initialize cashflow matrix
    CF = np.maximum(K - S[:,-1], 0)
    
    # Backwards induction
    for i in range(M-1, 0, -1):
        in_the_money = S[:,i] < K
        X = S[in_the_money,i]
        Y = CF[in_the_money] * np.exp(-r*dt)
        
        # Legendre polynomial basis
        X_norm = 2*(X - K)/(K) - 1  # Normalize to [-1,1]
        basis = legendre.legvander(X_norm, degree)
        
        # Least squares regression
        coeffs = np.linalg.lstsq(basis, Y, rcond=None)[0]
        continuation = legendre.legval(X_norm, coeffs)
        
        # Exercise decision
        exercise = np.maximum(K - X, 0)
        exercise_idx = exercise > continuation
        
        CF[in_the_money] = np.where(exercise_idx, exercise, Y)
    
    # Discount final cashflows
    price = np.exp(-r*dt) * np.mean(CF)
    return price
\end{minted}

\subsection{Regression Diagnostics}
\begin{table}[h]
\centering
\begin{tabular}{@{}lll@{}}
\toprule
Basis Degree & Price Estimate & Std Error \\
\midrule
2 & 4.478 & 0.012 \\
3 & 4.486 & 0.011 \\
4 & 4.483 & 0.011 \\
\bottomrule
\end{tabular}
\caption{Convergence with Basis Function Degree}
\end{table}

\appendix
\section{Proofs}
\subsection{Dynamic Programming Proof}
\label{app:dp_proof}
\begin{proof}
The proof follows by...
\end{proof}

\end{document}