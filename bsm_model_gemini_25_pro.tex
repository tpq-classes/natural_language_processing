\documentclass[12pt]{article}

\usepackage[margin=1in]{geometry}
\usepackage{amsmath}
\usepackage{amssymb}
\usepackage{amsthm}
\usepackage{hyperref}
\usepackage{bm} % For bold math symbols like Greeks

% Define theorem-like environments
\newtheorem{theorem}{Theorem}[section]
\newtheorem{lemma}[theorem]{Lemma}
\newtheorem{proposition}[theorem]{Proposition}
\newtheorem{corollary}[theorem]{Corollary}
\newtheorem{definition}[theorem]{Definition}

% Custom commands for clarity if needed (e.g., for partial derivatives)
\newcommand{\pderiv}[2]{\frac{\partial #1}{\partial #2}}
\newcommand{\pderivsq}[2]{\frac{\partial^2 #1}{\partial #2^2}}
\newcommand{\N}{\mathcal{N}} % Standard normal CDF
\newcommand{\n}{\phi} % Standard normal PDF (using \phi for PDF to avoid conflict with n variable)

\title{The Black-Scholes Option Pricing Model: A Comprehensive Introduction}
\author{Generated by AI}
\date{\today}

\begin{document}

\maketitle
\begin{abstract}
This document provides a comprehensive mathematical introduction to the Black-Scholes option pricing model. It covers the historical context, core assumptions, the derivation of the Black-Scholes-Merton partial differential equation, the derivation of the European call and put option pricing formulas, and an analysis of the option `Greeks'. All mathematical proofs and derivations are presented in full detail, aimed at readers with a solid mathematical background but who may be new to quantitative finance. The limitations of the model and a brief conclusion are also discussed.
\end{abstract}

\tableofcontents

\section{Introduction and Historical Context}

\subsection{What are Options?}
An option is a financial contract that gives the buyer (holder) the right, but not the obligation, to buy or sell an underlying asset at a specified price (the strike price) on or before a certain date (the expiration date).
\begin{itemize}
    \item \textbf{Call Option}: Gives the holder the right to buy the underlying asset.
    \item \textbf{Put Option}: Gives the holder the right to sell the underlying asset.
    \item \textbf{European Option}: Can only be exercised on the expiration date.
    \item \textbf{American Option}: Can be exercised at any time up to and including the expiration date.
\end{itemize}
The Black-Scholes model, in its original form, deals with European options.

\subsection{Significance of Option Pricing}
Pricing options accurately is crucial for traders, investors, and risk managers. Before the Black-Scholes model, there was no universally accepted method for valuing options. The model provided a theoretical framework that revolutionized derivatives markets.

\subsection{Historical Development}
The Black-Scholes model was first published in 1973 by Fischer Black and Myron Scholes in their paper ``The Pricing of Options and Corporate Liabilities'' in the Journal of Political Economy. Robert C. Merton independently extended their work and published ``Theory of Rational Option Pricing'' in the Bell Journal of Economics and Management Science, also in 1973. Merton's contribution was significant in providing a more rigorous mathematical derivation and showing the broad applicability of the underlying methodology.

Myron Scholes and Robert Merton were awarded the 1997 Nobel Memorial Prize in Economic Sciences for their work. Fischer Black had passed away in 1995 and was thus ineligible for the prize, but his contribution was acknowledged by the Nobel committee.

The model's development built upon earlier work in stochastic calculus, particularly by Kiyosi Itô, and theories of asset pricing under uncertainty.

\section{Core Assumptions of the Model}
The Black-Scholes model relies on a set of idealized assumptions about the market and the behavior of the underlying asset:

\begin{enumerate}
    \item \textbf{Stock Price Behavior}: The price of the underlying asset follows a Geometric Brownian Motion (GBM) with constant drift $\mu$ and constant volatility $\sigma$. This implies that log-returns of the stock are normally distributed.
    \item \textbf{Constant Parameters}: The risk-free interest rate $r$ and the volatility $\sigma$ of the underlying asset are constant and known over the life of the option.
    \item \textbf{No Dividends}: The underlying stock does not pay dividends during the life of the option. (The model can be adjusted for known discrete or continuous dividends.)
    \item \textbf{No Transaction Costs or Taxes}: There are no commissions or other costs associated with buying or selling the asset or the option, and no taxes.
    \item \textbf{Perfectly Liquid Markets and Continuous Trading}: It is possible to buy or sell any amount of the asset (including fractional amounts) at any time without affecting its price. Trading can occur continuously.
    \item \textbf{European Options}: The option can only be exercised at expiration.
    \item \textbf{No Arbitrage Opportunities}: There are no possibilities for risk-free profit. This is a fundamental assumption that underpins the derivation of the model.
    \item \textbf{Short Selling}: Short selling of the underlying asset is permitted and the proceeds are available to the seller. Borrowing and lending can be done at the risk-free rate.
\end{enumerate}
It is important to note that many of these assumptions are simplifications of reality, which leads to limitations of the model.

\section{The Black-Scholes-Merton Differential Equation}
The Black-Scholes-Merton Partial Differential Equation (PDE) is a cornerstone of modern financial theory. It describes how the price of an option $V(S,t)$ changes with respect to the price of the underlying asset $S$ and time $t$.

\subsection{Stochastic Process for Stock Price}
The model assumes that the stock price $S_t$ follows a Geometric Brownian Motion (GBM):
\begin{equation} \label{eq:gbm}
dS_t = \mu S_t dt + \sigma S_t dW_t
\end{equation}
where:
\begin{itemize}
    \item $S_t$ is the stock price at time $t$.
    \item $\mu$ is the expected rate of return on the stock (drift coefficient).
    \item $\sigma$ is the volatility of the stock's returns (diffusion coefficient).
    \item $dt$ is an infinitesimal time increment.
    \item $dW_t$ is an increment of a Wiener process (or standard Brownian motion). $dW_t$ is a random variable such that $dW_t = Z\sqrt{dt}$, where $Z \sim \mathcal{N}(0,1)$ (standard normal distribution). This implies $E[dW_t] = 0$ and $\mathrm{Var}(dW_t) = E[(dW_t)^2] = dt$.
\end{itemize}

\subsection{Itô's Lemma}
To understand how the option price $V(S,t)$ changes, we need Itô's Lemma, which is a rule for differentiating a function of a stochastic process.

\begin{lemma}[Itô's Lemma]
Let $S_t$ be a stochastic process following $dS_t = a(S_t, t) dt + b(S_t, t) dW_t$. Let $V(S,t)$ be a twice-differentiable function with respect to $S$ and once-differentiable with respect to $t$. Then the differential $dV$ is given by:
\begin{equation} \label{eq:ito_lemma}
dV = \left( \pderiv{V}{t} + a(S_t,t) \pderiv{V}{S} + \frac{1}{2} b(S_t,t)^2 \pderivsq{V}{S} \right) dt + b(S_t,t) \pderiv{V}{S} dW_t
\end{equation}
\end{lemma}

\begin{proof}[Derivation Sketch of Itô's Lemma]
Consider a Taylor expansion of $V(S,t)$:
$$ dV \approx \pderiv{V}{t} dt + \pderiv{V}{S} dS + \frac{1}{2} \pderivsq{V}{S} (dS)^2 + \frac{1}{2} \pderivsq{V}{t} (dt)^2 + \pderiv{^2 V}{S \partial t} dS dt + \dots $$
We substitute $dS = a dt + b dW_t$.
The key insight from stochastic calculus relates to the quadratic variation of the Wiener process: $(dW_t)^2 \approx dt$. Other terms are of lower order: $(dt)^2 \approx 0$, $dt dW_t \approx 0$.
So, $(dS)^2 = (a dt + b dW_t)^2 = a^2 (dt)^2 + b^2 (dW_t)^2 + 2ab dt dW_t$.
Keeping terms up to order $dt$:
$(dS)^2 \approx b^2 dt$.
Substituting $dS$ and $(dS)^2$ into the Taylor expansion and ignoring terms of order higher than $dt$:
$$ dV \approx \pderiv{V}{t} dt + \pderiv{V}{S} (a dt + b dW_t) + \frac{1}{2} \pderivsq{V}{S} (b^2 dt) $$
Rearranging terms:
$$ dV = \left( \pderiv{V}{t} + a \pderiv{V}{S} + \frac{1}{2} b^2 \pderivsq{V}{S} \right) dt + b \pderiv{V}{S} dW_t $$
For our stock price process $dS_t = \mu S_t dt + \sigma S_t dW_t$, we have $a(S_t,t) = \mu S_t$ and $b(S_t,t) = \sigma S_t$.
Applying Itô's Lemma to $V(S,t)$:
\begin{equation} \label{eq:dV_ito}
dV = \left( \pderiv{V}{t} + \mu S \pderiv{V}{S} + \frac{1}{2} \sigma^2 S^2 \pderivsq{V}{S} \right) dt + \sigma S \pderiv{V}{S} dW_t
\end{equation}
\end{proof}

\subsection{Constructing a Risk-Free Portfolio (Delta Hedging)}
The derivation of the Black-Scholes PDE relies on forming a risk-free portfolio consisting of one option and a certain number of shares of the underlying stock.
Let $\Pi$ be the value of a portfolio consisting of:
\begin{itemize}
    \item One unit of the option (value $V$).
    \item Short position in $\Delta$ units of the underlying stock (value $-\Delta S$).
\end{itemize}
So, the value of the portfolio is $\Pi = V - \Delta S$.
The change in the value of this portfolio over a small time interval $dt$ is:
$$ d\Pi = dV - \Delta dS $$
Substitute $dV$ from Equation \eqref{eq:dV_ito} and $dS$ from Equation \eqref{eq:gbm}:
$$ d\Pi = \left[ \left( \pderiv{V}{t} + \mu S \pderiv{V}{S} + \frac{1}{2} \sigma^2 S^2 \pderivsq{V}{S} \right) dt + \sigma S \pderiv{V}{S} dW_t \right] - \Delta (\mu S dt + \sigma S dW_t) $$
Collect terms with $dt$ and $dW_t$:
$$ d\Pi = \left( \pderiv{V}{t} + \mu S \pderiv{V}{S} + \frac{1}{2} \sigma^2 S^2 \pderivsq{V}{S} - \Delta \mu S \right) dt + \left( \sigma S \pderiv{V}{S} - \Delta \sigma S \right) dW_t $$
To make the portfolio risk-free, we need to eliminate the stochastic term involving $dW_t$. This is achieved by choosing $\Delta$ such that the coefficient of $dW_t$ is zero:
$$ \sigma S \pderiv{V}{S} - \Delta \sigma S = 0 $$
This implies:
\begin{equation} \label{eq:delta_choice}
\Delta = \pderiv{V}{S}
\end{equation}
This strategy is known as delta hedging. The quantity $\Delta$ is the option's delta, representing the sensitivity of the option price to changes in the stock price.
With this choice of $\Delta$, the change in the portfolio value becomes:
$$ d\Pi = \left( \pderiv{V}{t} + \mu S \pderiv{V}{S} + \frac{1}{2} \sigma^2 S^2 \pderivsq{V}{S} - \left(\pderiv{V}{S}\right) \mu S \right) dt $$
$$ d\Pi = \left( \pderiv{V}{t} + \frac{1}{2} \sigma^2 S^2 \pderivsq{V}{S} \right) dt $$
Notice that the drift term $\mu S \pderiv{V}{S}$ cancels out with $-\Delta \mu S$. The portfolio is now instantaneously risk-free; its change $d\Pi$ is deterministic.

\subsection{No Arbitrage Argument}
By the no-arbitrage principle, a risk-free portfolio must earn the risk-free interest rate $r$. If it earned more, investors could make unlimited profits by borrowing at $r$ and investing in the portfolio. If it earned less, investors could short the portfolio and invest the proceeds at $r$.
Therefore, the change in the portfolio value $d\Pi$ must also be equal to the risk-free return on the portfolio value $\Pi$:
$$ d\Pi = r \Pi dt $$
Substitute $\Pi = V - \Delta S = V - \pderiv{V}{S} S$:
$$ d\Pi = r \left( V - S \pderiv{V}{S} \right) dt $$
Equating the two expressions for $d\Pi$:
$$ \left( \pderiv{V}{t} + \frac{1}{2} \sigma^2 S^2 \pderivsq{V}{S} \right) dt = r \left( V - S \pderiv{V}{S} \right) dt $$
Dividing by $dt$ (assuming $dt \neq 0$) and rearranging the terms, we obtain the Black-Scholes-Merton Partial Differential Equation:
\begin{equation} \label{eq:bsm_pde}
\pderiv{V}{t} + r S \pderiv{V}{S} + \frac{1}{2} \sigma^2 S^2 \pderivsq{V}{S} - rV = 0
\end{equation}
This PDE governs the price $V(S,t)$ of any derivative security that depends on a non-dividend-paying stock $S$ following GBM, under the model's assumptions.
Notably, the stock's expected rate of return $\mu$ does not appear in the equation. This is a crucial result: the option price does not depend on investors' risk preferences or their expectations of the stock's future return, only on its volatility $\sigma$.

Explanation of terms in the BSM PDE:
\begin{itemize}
    \item $\pderiv{V}{t}$: Theta ($\Theta$), represents time decay of the option's value.
    \item $r S \pderiv{V}{S}$: Term related to Delta ($\Delta = \pderiv{V}{S}$), representing the change in option value due to change in stock price, financed at risk-free rate.
    \item $\frac{1}{2} \sigma^2 S^2 \pderivsq{V}{S}$: Term related to Gamma ($\Gamma = \pderivsq{V}{S}$), representing the convexity of the option value with respect to stock price, scaled by volatility.
    \item $-rV$: Discounting term, reflecting that the option must earn the risk-free rate on its own value if it were risk-free.
\end{itemize}

\section{The Black-Scholes Formula for European Call and Put Options}
To find the specific price of a European call or put option, we need to solve the BSM PDE \eqref{eq:bsm_pde} subject to appropriate boundary conditions. The boundary condition for a European option is its payoff at expiration time $T$.

\subsection{Risk-Neutral Pricing}
The BSM PDE does not contain $\mu$, the expected return on the stock. This implies that the PDE holds in any world, regardless of risk preferences. A common technique to solve such PDEs in finance is to assume a ``risk-neutral world". In this hypothetical world:
\begin{enumerate}
    \item All investors are risk-neutral.
    \item The expected return on all assets (including the stock) is the risk-free interest rate $r$.
\end{enumerate}
Under risk-neutrality, the stock price process becomes:
\begin{equation} \label{eq:gbm_risk_neutral}
dS_t = r S_t dt + \sigma S_t dW_t^Q
\end{equation}
where $dW_t^Q$ is a Wiener process under the risk-neutral measure $Q$.
The Feynman-Kac theorem provides a link between PDEs like the BSM equation and expectations of discounted future payoffs. It states that the solution $V(S_t,t)$ to the BSM PDE is the expected present value of the option's payoff at expiration $T$, computed under the risk-neutral measure $Q$:
\begin{equation} \label{eq:feynman_kac}
V(S_t, t) = E_Q \left[ e^{-r(T-t)} \text{Payoff}(S_T) | \mathcal{F}_t \right]
\end{equation}
where $E_Q[\cdot | \mathcal{F}_t]$ is the expectation under the risk-neutral measure conditional on information available at time $t$. Let $\tau = T-t$ be the time to maturity.

\subsection{Derivation of the Call Option Formula}
For a European call option, the payoff at expiration $T$ is $\max(S_T - K, 0)$, where $K$ is the strike price.
The call option price $C(S_t, t)$ is:
\begin{equation} \label{eq:call_expectation}
C(S_t, t) = e^{-r\tau} E_Q[\max(S_T - K, 0) | S_t]
\end{equation}
Under the risk-neutral GBM (Equation \eqref{eq:gbm_risk_neutral}), the solution for $S_T$ given $S_t$ is:
\begin{equation} \label{eq:ST_lognormal}
S_T = S_t \exp\left( (r - \frac{1}{2}\sigma^2)\tau + \sigma \sqrt{\tau} Z \right)
\end{equation}
where $Z \sim \mathcal{N}(0,1)$ is a standard normal random variable. This means $\ln(S_T/S_t)$ is normally distributed with mean $(r - \frac{1}{2}\sigma^2)\tau$ and variance $\sigma^2\tau$.

The expectation in Equation \eqref{eq:call_expectation} becomes an integral:
$$ C(S_t, t) = e^{-r\tau} \int_{-\infty}^{\infty} \max(S_t e^{(r - \frac{1}{2}\sigma^2)\tau + \sigma \sqrt{\tau} z} - K, 0) \frac{1}{\sqrt{2\pi}} e^{-z^2/2} dz $$
The term $\max(S_T - K, 0)$ is non-zero only if $S_T > K$. We need to find the values of $z$ for which this condition holds:
$$ S_t e^{(r - \frac{1}{2}\sigma^2)\tau + \sigma \sqrt{\tau} z} > K $$
$$ e^{(r - \frac{1}{2}\sigma^2)\tau + \sigma \sqrt{\tau} z} > \frac{K}{S_t} $$
$$ (r - \frac{1}{2}\sigma^2)\tau + \sigma \sqrt{\tau} z > \ln\left(\frac{K}{S_t}\right) $$
$$ \sigma \sqrt{\tau} z > \ln\left(\frac{K}{S_t}\right) - (r - \frac{1}{2}\sigma^2)\tau $$
$$ z > \frac{\ln(K/S_t) - (r - \frac{1}{2}\sigma^2)\tau}{\sigma \sqrt{\tau}} = \frac{-\left[\ln(S_t/K) + (r - \frac{1}{2}\sigma^2)\tau\right]}{\sigma \sqrt{\tau}} $$
Let's define $d_2$ as:
\begin{equation} \label{eq:d2_def}
d_2 = \frac{\ln(S_t/K) + (r - \frac{1}{2}\sigma^2)\tau}{\sigma \sqrt{\tau}}
\end{equation}
So the condition for $S_T > K$ is $z > -d_2$. The integral is thus from $-d_2$ to $\infty$.
$$ C(S_t, t) = e^{-r\tau} \left[ \int_{-d_2}^{\infty} S_t e^{(r - \frac{1}{2}\sigma^2)\tau + \sigma \sqrt{\tau} z} \frac{1}{\sqrt{2\pi}} e^{-z^2/2} dz - K \int_{-d_2}^{\infty} \frac{1}{\sqrt{2\pi}} e^{-z^2/2} dz \right] $$
Let $n(x) = \frac{1}{\sqrt{2\pi}} e^{-x^2/2}$ be the PDF of the standard normal distribution, and $N(x) = \int_{-\infty}^x n(u)du$ be its CDF.
The second integral is:
$$ \int_{-d_2}^{\infty} n(z) dz = 1 - N(-d_2) = N(d_2) $$
So the second term is $-K e^{-r\tau} N(d_2)$.

Now consider the first integral, $I_1$:
$$ I_1 = \int_{-d_2}^{\infty} S_t e^{(r - \frac{1}{2}\sigma^2)\tau + \sigma \sqrt{\tau} z} n(z) dz $$
$$ I_1 = S_t e^{(r - \frac{1}{2}\sigma^2)\tau} \int_{-d_2}^{\infty} \frac{1}{\sqrt{2\pi}} e^{\sigma \sqrt{\tau} z - z^2/2} dz $$
The exponent is $\sigma \sqrt{\tau} z - z^2/2 = -\frac{1}{2}(z^2 - 2\sigma \sqrt{\tau} z)$.
Complete the square: $z^2 - 2\sigma \sqrt{\tau} z = (z - \sigma \sqrt{\tau})^2 - \sigma^2 \tau$.
So, $e^{\sigma \sqrt{\tau} z - z^2/2} = e^{-\frac{1}{2}(z - \sigma \sqrt{\tau})^2 + \frac{1}{2}\sigma^2 \tau} = e^{-\frac{1}{2}(z - \sigma \sqrt{\tau})^2} e^{\frac{1}{2}\sigma^2 \tau}$.
$$ I_1 = S_t e^{(r - \frac{1}{2}\sigma^2)\tau} e^{\frac{1}{2}\sigma^2 \tau} \int_{-d_2}^{\infty} \frac{1}{\sqrt{2\pi}} e^{-\frac{1}{2}(z - \sigma \sqrt{\tau})^2} dz $$
$$ I_1 = S_t e^{r\tau} \int_{-d_2}^{\infty} \frac{1}{\sqrt{2\pi}} e^{-\frac{1}{2}(z - \sigma \sqrt{\tau})^2} dz $$
Let $y = z - \sigma \sqrt{\tau}$. Then $dz = dy$. The lower limit of integration becomes:
$z = -d_2 \implies y = -d_2 - \sigma \sqrt{\tau}$.
Substitute $d_2 = \frac{\ln(S_t/K) + (r - \frac{1}{2}\sigma^2)\tau}{\sigma \sqrt{\tau}}$:
$$ y = -\frac{\ln(S_t/K) + (r - \frac{1}{2}\sigma^2)\tau}{\sigma \sqrt{\tau}} - \sigma \sqrt{\tau} $$
$$ y = -\frac{\ln(S_t/K) + (r - \frac{1}{2}\sigma^2)\tau + \sigma^2 \tau}{\sigma \sqrt{\tau}} = -\frac{\ln(S_t/K) + (r + \frac{1}{2}\sigma^2)\tau}{\sigma \sqrt{\tau}} $$
Let's define $d_1$ as:
\begin{equation} \label{eq:d1_def}
d_1 = \frac{\ln(S_t/K) + (r + \frac{1}{2}\sigma^2)\tau}{\sigma \sqrt{\tau}}
\end{equation}
So the lower limit is $y = -d_1$.
The integral becomes $\int_{-d_1}^{\infty} \frac{1}{\sqrt{2\pi}} e^{-y^2/2} dy = N(d_1)$.
Thus, $I_1 = S_t e^{r\tau} N(d_1)$.
The first term of $C(S_t,t)$ is $e^{-r\tau} I_1 = e^{-r\tau} S_t e^{r\tau} N(d_1) = S_t N(d_1)$.
Combining the terms, the Black-Scholes formula for a European call option is:
\begin{equation} \label{eq:bs_call_formula}
C(S_t, t) = S_t N(d_1) - K e^{-r\tau} N(d_2)
\end{equation}
where $\tau = T-t$ is the time to maturity, and
\begin{align}
d_1 &= \frac{\ln(S_t/K) + (r + \frac{1}{2}\sigma^2)\tau}{\sigma \sqrt{\tau}} \label{eq:d1_formula} \\
d_2 &= \frac{\ln(S_t/K) + (r - \frac{1}{2}\sigma^2)\tau}{\sigma \sqrt{\tau}} = d_1 - \sigma \sqrt{\tau} \label{eq:d2_formula}
\end{align}
And $N(x)$ is the cumulative distribution function (CDF) of the standard normal distribution.

\textbf{Interpretation of components}:
\begin{itemize}
    \item $S_t N(d_1)$: The expected present value of receiving the stock if the option finishes in-the-money. $N(d_1)$ can be interpreted as the probability of $S_T > K$ in a measure where the stock price is the numeraire, multiplied by a discount factor, or as $E_Q[S_T \mathbf{1}_{S_T>K}] / S_t$.
    \item $K e^{-r\tau} N(d_2)$: The present value of paying the strike price if the option finishes in-the-money. $N(d_2)$ is the risk-neutral probability that the option will be exercised, $P_Q(S_T > K)$.
\end{itemize}

\subsection{Put-Call Parity and the Put Option Formula}
Put-Call Parity establishes a relationship between the price of a European call option and a European put option with the same strike price $K$ and expiration date $T$.

\begin{theorem}[Put-Call Parity]
For European options on a non-dividend-paying stock:
\begin{equation} \label{eq:put_call_parity}
C(S_t, t) - P(S_t, t) = S_t - K e^{-r\tau}
\end{equation}
where $P(S_t,t)$ is the price of the European put option.
\end{theorem}
\begin{proof}
Consider two portfolios constructed at time $t$:
\begin{itemize}
    \item \textbf{Portfolio A}: One call option ($C$) and cash amounting to $K e^{-r\tau}$ (invested at the risk-free rate).
    Value at $t$: $V_A(t) = C + K e^{-r\tau}$.
    \item \textbf{Portfolio B}: One put option ($P$) and one share of the underlying stock ($S_t$).
    Value at $t$: $V_B(t) = P + S_t$.
\end{itemize}
At expiration $T$, the cash in Portfolio A will have grown to $K$.
The values of the portfolios at expiration $T$ are:
\begin{itemize}
    \item If $S_T > K$:
        Portfolio A: $C_T = S_T - K$. Cash = $K$. Total $V_A(T) = (S_T - K) + K = S_T$.
        Portfolio B: $P_T = 0$. Stock = $S_T$. Total $V_B(T) = 0 + S_T = S_T$.
    \item If $S_T \le K$:
        Portfolio A: $C_T = 0$. Cash = $K$. Total $V_A(T) = 0 + K = K$.
        Portfolio B: $P_T = K - S_T$. Stock = $S_T$. Total $V_B(T) = (K - S_T) + S_T = K$.
\end{itemize}
In both cases, $V_A(T) = V_B(T)$. Since the portfolios have identical payoffs at expiration $T$, by the no-arbitrage principle, they must have the same value at time $t$:
$C + K e^{-r\tau} = P + S_t$.
Rearranging gives the Put-Call Parity formula.
\end{proof}

Using Put-Call Parity, we can derive the formula for a European put option:
$$ P(S_t, t) = C(S_t, t) - S_t + K e^{-r\tau} $$
Substitute the Black-Scholes call formula (Equation \eqref{eq:bs_call_formula}):
$$ P(S_t, t) = (S_t N(d_1) - K e^{-r\tau} N(d_2)) - S_t + K e^{-r\tau} $$
$$ P(S_t, t) = S_t (N(d_1) - 1) + K e^{-r\tau} (1 - N(d_2)) $$
Using the property of the standard normal CDF that $N(x) - 1 = -N(-x)$ and $1 - N(x) = N(-x)$:
$$ N(d_1) - 1 = -N(-d_1) $$
$$ 1 - N(d_2) = N(-d_2) $$
So, the Black-Scholes formula for a European put option is:
\begin{equation} \label{eq:bs_put_formula}
P(S_t, t) = K e^{-r\tau} N(-d_2) - S_t N(-d_1)
\end{equation}
where $d_1$ and $d_2$ are the same as defined for the call option.

\section{The `Greeks'}
The `Greeks' are measures of the sensitivity of an option's price to changes in underlying parameters. They are the partial derivatives of the option price formula. Let $V$ denote the option price (either $C$ or $P$). $\tau = T-t$.

For the derivations, we will need the PDF of the standard normal distribution, $n(x) = \frac{1}{\sqrt{2\pi}}e^{-x^2/2}$.
Note that $\pderiv{N(x)}{y} = n(x) \pderiv{x}{y}$.
A key identity used in deriving the Greeks is:
\begin{equation} \label{eq:greek_identity}
S_t n(d_1) = K e^{-r\tau} n(d_2)
\end{equation}
\begin{proof}
$n(d_1)/n(d_2) = e^{-(d_1^2-d_2^2)/2}$.
$d_1^2 - d_2^2 = (d_1-d_2)(d_1+d_2) = (\sigma\sqrt{\tau})(d_1+d_2)$.
$d_1+d_2 = \frac{2\ln(S_t/K) + 2r\tau}{\sigma\sqrt{\tau}}$.
So $d_1^2 - d_2^2 = (\sigma\sqrt{\tau}) \frac{2\ln(S_t/K) + 2r\tau}{\sigma\sqrt{\tau}} = 2(\ln(S_t/K) + r\tau)$.
Then $n(d_1)/n(d_2) = e^{-(\ln(S_t/K) + r\tau)} = e^{-\ln(S_t/K)} e^{-r\tau} = (K/S_t) e^{-r\tau}$.
Thus $S_t n(d_1) = K e^{-r\tau} n(d_2)$.
\end{proof}

\subsection{Delta ($\Delta$)}
Delta measures the rate of change of the option price with respect to a change in the underlying asset's price.
$$ \Delta = \pderiv{V}{S_t} $$
We need $\pderiv{d_1}{S_t} = \frac{1}{S_t \sigma \sqrt{\tau}}$ and $\pderiv{d_2}{S_t} = \frac{1}{S_t \sigma \sqrt{\tau}}$.

\textbf{Call Delta ($\Delta_C$)}:
$$ \Delta_C = \pderiv{}{S_t} (S_t N(d_1) - K e^{-r\tau} N(d_2)) $$
$$ \Delta_C = N(d_1) + S_t n(d_1) \pderiv{d_1}{S_t} - K e^{-r\tau} n(d_2) \pderiv{d_2}{S_t} $$
$$ \Delta_C = N(d_1) + S_t n(d_1) \frac{1}{S_t \sigma \sqrt{\tau}} - K e^{-r\tau} n(d_2) \frac{1}{S_t \sigma \sqrt{\tau}} $$
$$ \Delta_C = N(d_1) + \frac{1}{S_t \sigma \sqrt{\tau}} [S_t n(d_1) - K e^{-r\tau} n(d_2)] $$
Using identity \eqref{eq:greek_identity}, the term in the square brackets is zero.
\begin{equation}
\Delta_C = N(d_1)
\end{equation}
Interpretation: For a small change in stock price, the call option price changes by approximately $N(d_1)$ times that change. $0 < N(d_1) < 1$.

\textbf{Put Delta ($\Delta_P$)}:
Using Put-Call Parity $P = C - S_t + K e^{-r\tau}$:
$$ \Delta_P = \pderiv{P}{S_t} = \pderiv{C}{S_t} - \pderiv{S_t}{S_t} + \pderiv{(K e^{-r\tau})}{S_t} = \Delta_C - 1 $$
\begin{equation}
\Delta_P = N(d_1) - 1 = -N(-d_1)
\end{equation}
Interpretation: Put delta is negative, $-1 < N(d_1)-1 < 0$. As stock price increases, put value decreases.

\subsection{Gamma ($\Gamma$)}
Gamma measures the rate of change of Delta with respect to a change in the underlying asset's price.
$$ \Gamma = \pderivsq{V}{S_t} = \pderiv{\Delta}{S_t} $$
\textbf{Call/Put Gamma ($\Gamma_C = \Gamma_P$)}:
$$ \Gamma_C = \pderiv{\Delta_C}{S_t} = \pderiv{N(d_1)}{S_t} = n(d_1) \pderiv{d_1}{S_t} $$
\begin{equation}
\Gamma_C = n(d_1) \frac{1}{S_t \sigma \sqrt{\tau}} = \frac{n(d_1)}{S_t \sigma \sqrt{\tau}}
\end{equation}
Since $\Delta_P = \Delta_C - 1$, $\Gamma_P = \pderiv{(\Delta_C-1)}{S_t} = \Gamma_C$.
\begin{equation}
\Gamma = \frac{n(d_1)}{S_t \sigma \sqrt{\tau}}
\end{equation}
Interpretation: Gamma measures the convexity of the option price. It is highest for at-the-money options close to expiration. Important for managing delta-hedged portfolios.

\subsection{Vega ($\mathcal{V}$ or $\nu$)}
Vega measures the sensitivity of the option price to changes in the volatility of the underlying asset. (Note: Vega is not an actual Greek letter).
$$ \mathcal{V} = \pderiv{V}{\sigma} $$
We need $\pderiv{d_1}{\sigma}$ and $\pderiv{d_2}{\sigma}$.
$d_1 = \frac{\ln(S_t/K) + r\tau}{\sigma\sqrt{\tau}} + \frac{1}{2}\sigma\sqrt{\tau}$.
$\pderiv{d_1}{\sigma} = -\frac{\ln(S_t/K) + r\tau}{\sigma^2\sqrt{\tau}} + \frac{1}{2}\sqrt{\tau} = -\frac{1}{\sigma} \left( d_1 - \frac{1}{2}\sigma\sqrt{\tau} \right) + \frac{1}{2}\sqrt{\tau} = -\frac{d_1}{\sigma} + \sqrt{\tau}$.
$d_2 = d_1 - \sigma\sqrt{\tau}$.
$\pderiv{d_2}{\sigma} = \pderiv{d_1}{\sigma} - \sqrt{\tau} = (-\frac{d_1}{\sigma} + \sqrt{\tau}) - \sqrt{\tau} = -\frac{d_1}{\sigma}$.

\textbf{Call Vega ($\mathcal{V}_C$)}:
$$ \mathcal{V}_C = \pderiv{C}{\sigma} = S_t n(d_1) \pderiv{d_1}{\sigma} - K e^{-r\tau} n(d_2) \pderiv{d_2}{\sigma} $$
$$ \mathcal{V}_C = S_t n(d_1) \left(-\frac{d_1}{\sigma} + \sqrt{\tau}\right) - K e^{-r\tau} n(d_2) \left(-\frac{d_1}{\sigma}\right) $$
Using identity \eqref{eq:greek_identity}, $K e^{-r\tau} n(d_2) = S_t n(d_1)$:
$$ \mathcal{V}_C = S_t n(d_1) \left(-\frac{d_1}{\sigma} + \sqrt{\tau}\right) - S_t n(d_1) \left(-\frac{d_1}{\sigma}\right) $$
$$ \mathcal{V}_C = S_t n(d_1) \left(-\frac{d_1}{\sigma} + \sqrt{\tau} + \frac{d_1}{\sigma}\right) $$
\begin{equation}
\mathcal{V}_C = S_t n(d_1) \sqrt{\tau}
\end{equation}
\textbf{Put Vega ($\mathcal{V}_P$)}:
Since $P = C - S_t + K e^{-r\tau}$, and $S_t, K, r, \tau$ do not depend on $\sigma$ (as a parameter for differentiation):
$$ \mathcal{V}_P = \pderiv{P}{\sigma} = \pderiv{C}{\sigma} = \mathcal{V}_C $$
\begin{equation}
\mathcal{V} = S_t n(d_1) \sqrt{\tau}
\end{equation}
Interpretation: Vega is positive for both calls and puts. Higher volatility increases option prices.

\subsection{Theta ($\Theta$)}
Theta measures the sensitivity of the option price to the passage of time (time decay). It is defined as $-\pderiv{V}{t}$. Since $\tau = T-t$, $dt = -d\tau$, so $\Theta = \pderiv{V}{\tau}$.
The prompt specifies $\Theta = -\pderiv{V}{t}$.
The BSM PDE is $\pderiv{V}{t} + rS_t\pderiv{V}{S_t} + \frac{1}{2}\sigma^2 S_t^2 \pderivsq{V}{S_t} - rV = 0$.
So, $\pderiv{V}{t} = rV - rS_t\Delta - \frac{1}{2}\sigma^2 S_t^2\Gamma$.
Therefore, $\Theta = -\pderiv{V}{t} = -rV + rS_t\Delta + \frac{1}{2}\sigma^2 S_t^2\Gamma$.

\textbf{Call Theta ($\Theta_C$)}:
$$ \Theta_C = -r C + rS_t \Delta_C + \frac{1}{2}\sigma^2 S_t^2 \Gamma_C $$
$$ \Theta_C = -r(S_t N(d_1) - K e^{-r\tau}N(d_2)) + rS_t N(d_1) + \frac{1}{2}\sigma^2 S_t^2 \frac{n(d_1)}{S_t \sigma \sqrt{\tau}} $$
$$ \Theta_C = -rS_t N(d_1) + rK e^{-r\tau}N(d_2) + rS_t N(d_1) + \frac{S_t \sigma n(d_1)}{2 \sqrt{\tau}} $$
\begin{equation}
\Theta_C = \frac{S_t \sigma n(d_1)}{2 \sqrt{\tau}} + rK e^{-r\tau}N(d_2)
\end{equation}
The prompt asks for $\Theta = -\pderiv{V}{t}$. My calculation directly yields $\pderiv{V}{t}$.
So, $\Theta_C^{\text{prompt def}} = - \left( \frac{S_t \sigma n(d_1)}{2 \sqrt{\tau}} + rK e^{-r\tau}N(d_2) \right)$.
Theta is usually negative for long option positions, representing the loss in value as time passes.

\textbf{Put Theta ($\Theta_P$)}:
Using the same definition $\Theta_P = -\pderiv{P}{t}$:
$$ \Theta_P = -r P + rS_t \Delta_P + \frac{1}{2}\sigma^2 S_t^2 \Gamma_P $$
$$ \Theta_P = -r(K e^{-r\tau}N(-d_2) - S_t N(-d_1)) + rS_t (N(d_1)-1) + \frac{1}{2}\sigma^2 S_t^2 \frac{n(d_1)}{S_t \sigma \sqrt{\tau}} $$
$$ \Theta_P = -rK e^{-r\tau}N(-d_2) + rS_t N(-d_1) + rS_t N(d_1) - rS_t + \frac{S_t \sigma n(d_1)}{2 \sqrt{\tau}} $$
Using $N(-d_1) + N(d_1) = 1$ is not useful here.
$rS_t (N(-d_1) + N(d_1) - 1) = rS_t (1 - 1) = 0$.
No, $N(-d_1) + (N(d_1)-1) = N(-d_1) - N(-d_1) = 0$. Correct.
This leads to:
$$ \Theta_P^{\text{prompt def}} = - \left( \frac{S_t \sigma n(d_1)}{2 \sqrt{\tau}} - rK e^{-r\tau}N(-d_2) \right) $$
So if Theta is time decay ($-\frac{\partial V}{\partial t}$):
\begin{align}
\Theta_C &= -\frac{S_t n(d_1) \sigma}{2\sqrt{\tau}} - r K e^{-r\tau} N(d_2) \\
\Theta_P &= -\frac{S_t n(d_1) \sigma}{2\sqrt{\tau}} + r K e^{-r\tau} N(-d_2)
\end{align}
Interpretation: Theta is generally negative for options, meaning their value erodes as expiry approaches, ceteris paribus. For certain deep in-the-money puts, Theta can be positive.

\subsection{Rho ($\rho$)}
Rho measures the sensitivity of the option price to changes in the risk-free interest rate.
$$ \rho = \pderiv{V}{r} $$
We need $\pderiv{d_1}{r}$ and $\pderiv{d_2}{r}$.
$d_1 = \frac{\ln(S_t/K) + (r + \frac{1}{2}\sigma^2)\tau}{\sigma \sqrt{\tau}}$.
$\pderiv{d_1}{r} = \frac{\tau}{\sigma\sqrt{\tau}} = \frac{\sqrt{\tau}}{\sigma}$.
$d_2 = d_1 - \sigma\sqrt{\tau}$.
$\pderiv{d_2}{r} = \pderiv{d_1}{r} = \frac{\sqrt{\tau}}{\sigma}$.

\textbf{Call Rho ($\rho_C$)}:
$$ \rho_C = \pderiv{C}{r} = S_t n(d_1) \pderiv{d_1}{r} - K (-\tau) e^{-r\tau} N(d_2) - K e^{-r\tau} n(d_2) \pderiv{d_2}{r} $$
$$ \rho_C = S_t n(d_1) \frac{\sqrt{\tau}}{\sigma} + K \tau e^{-r\tau} N(d_2) - K e^{-r\tau} n(d_2) \frac{\sqrt{\tau}}{\sigma} $$
Using identity \eqref{eq:greek_identity}, $S_t n(d_1) = K e^{-r\tau} n(d_2)$:
The first and third terms: $\frac{\sqrt{\tau}}{\sigma} [S_t n(d_1) - K e^{-r\tau} n(d_2)] = 0$.
\begin{equation}
\rho_C = K \tau e^{-r\tau} N(d_2)
\end{equation}
Interpretation: Rho is positive for calls. Higher interest rates increase call prices (because the present value of the strike price $K$ to be paid in the future decreases).

\textbf{Put Rho ($\rho_P$)}:
$$ P = K e^{-r\tau} N(-d_2) - S_t N(-d_1) $$
$$ \rho_P = \pderiv{P}{r} = K(-\tau)e^{-r\tau}N(-d_2) + K e^{-r\tau}(-n(d_2))(-\pderiv{d_2}{r}) - S_t (-n(d_1))(-\pderiv{d_1}{r}) $$
$$ \rho_P = -K\tau e^{-r\tau}N(-d_2) + K e^{-r\tau}n(d_2)\frac{\sqrt{\tau}}{\sigma} - S_t n(d_1)\frac{\sqrt{\tau}}{\sigma} $$
Again, the last two terms cancel due to identity \eqref{eq:greek_identity}.
\begin{equation}
\rho_P = -K \tau e^{-r\tau} N(-d_2)
\end{equation}
Interpretation: Rho is negative for puts. Higher interest rates decrease put prices.

\section{Limitations of the Model}
While groundbreaking, the Black-Scholes model has several limitations due to its simplifying assumptions:

\begin{enumerate}
    \item \textbf{Constant Volatility}: The assumption that volatility ($\sigma$) is constant is a major simplification. In reality, volatility is not constant; it changes over time and depends on the stock price level and strike price (this gives rise to the ``volatility smile'' and ``volatility skew'' observed in markets). Models like stochastic volatility models (e.g., Heston model) or local volatility models attempt to address this.
    \item \textbf{Constant Risk-Free Rate}: The risk-free rate is assumed constant. In reality, interest rates fluctuate.
    \item \textbf{No Dividends}: The original model assumes no dividends. While this can be adjusted for (e.g., by subtracting the present value of expected dividends from the stock price, or using a continuous dividend yield in the formulas), large, uncertain dividends can pose issues.
    \item \textbf{Log-Normal Distribution of Stock Prices}: GBM implies that stock returns are normally distributed (and prices are log-normally distributed). Empirical evidence shows that actual stock returns often exhibit ``fat tails'' (leptokurtosis) and skewness, meaning extreme price movements are more common than predicted by a normal distribution.
    \item \textbf{Continuous Trading and No Transaction Costs}: The assumption of continuous trading and no transaction costs allows for perfect delta hedging. In reality, trading is discrete, and transaction costs exist, making perfect hedging impossible.
    \item \textbf{European Options Only}: The model directly prices European options. Pricing American options, which can be exercised early, is more complex and generally requires numerical methods (though for American calls on non-dividend paying stocks, the price is the same as European calls).
    \item \textbf{No Arbitrage Violations}: While the model assumes no arbitrage, short-term arbitrage opportunities might exist due to market inefficiencies.
    \item \textbf{Liquidity Risk}: The model assumes perfect liquidity, meaning trades can be executed at any volume without impacting the market price. This is not always true, especially for large trades or illiquid assets.
\end{enumerate}
These limitations mean that the Black-Scholes prices are often used as a benchmark or starting point, and traders may make adjustments based on their own views or more sophisticated models.

\section{Conclusion}
The Black-Scholes-Merton model represents a monumental step forward in financial economics. It provided the first widely accepted and theoretically sound method for pricing options and other derivatives. Its derivation, based on the principles of no-arbitrage and dynamic hedging, introduced powerful mathematical tools (like stochastic calculus) to finance.

Despite its limitations, the model remains fundamental to the understanding of derivatives. It serves as a benchmark for more complex models and is still used extensively in practice, especially for its simplicity and the insights provided by the Greeks. The conceptual framework underlying the Black-Scholes model paved the way for the development of modern quantitative finance and risk management. Subsequent research has focused on relaxing its restrictive assumptions to create models that better reflect market realities, such as models incorporating stochastic volatility, jumps in asset prices, or transaction costs.

\end{document}