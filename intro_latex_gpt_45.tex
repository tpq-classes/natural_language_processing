\documentclass{article}

% No external packages, just standard LaTeX

\begin{document}

% Title, Author, and Date
\title{A Beginner-Friendly Introduction to LaTeX}
\author{Prepared by ChatGPT}
\date{\today}

\maketitle

\section{Introduction}

LaTeX is a document preparation system that allows you to create professional-looking documents easily. This guide introduces basic LaTeX features.

\section{Document Class and Structure}

Every LaTeX document begins with defining its document class.

\subsection{Example}

Rendered:

\textit{This document itself uses the \texttt{article} class.}

Code:

\begin{verbatim}
\documentclass{article}
\begin{document}
Your content here.
\end{document}
\end{verbatim}

\section{Sections and Subsections}

Sections structure your document.

\subsection{Example}

Rendered:

\textbf{Section Title}

\textit{Subsection Title}

Code:

\begin{verbatim}
\section{Section Title}
\subsection{Subsection Title}
\end{verbatim}

\section{Text Formatting}

You can format text as bold, italic, or underline easily.

\subsection{Example}

Rendered:

\textbf{Bold Text}, \textit{Italic Text}, \underline{Underlined Text}

Code:

\begin{verbatim}
\textbf{Bold Text}, \textit{Italic Text}, \underline{Underlined Text}
\end{verbatim}

\section{Lists}

Lists organize information.

\subsection{Itemized List Example}

Rendered:

\begin{itemize}
    \item First item
    \item Second item
    \item Third item
\end{itemize}

Code:

\begin{verbatim}
\begin{itemize}
    \item First item
    \item Second item
    \item Third item
\end{itemize}
\end{verbatim}

\subsection{Enumerated List Example}

Rendered:

\begin{enumerate}
    \item First item
    \item Second item
    \item Third item
\end{enumerate}

Code:

\begin{verbatim}
\begin{enumerate}
    \item First item
    \item Second item
    \item Third item
\end{enumerate}
\end{verbatim}

\section{Mathematical Expressions}

Math is one of LaTeX's strong suits.

\subsection{Inline Math Example}

Rendered:

Euler's formula is $e^{i\pi} + 1 = 0$.

Code:

\begin{verbatim}
Euler's formula is $e^{i\pi} + 1 = 0$.
\end{verbatim}

\subsection{Displayed Math Example}

Rendered:

\[
\int_0^\infty e^{-x} dx = 1
\]

Code:

\begin{verbatim}
\[
\int_0^\infty e^{-x} dx = 1
\]
\end{verbatim}

\section{Tables}

Tables present data clearly.

\subsection{Example}

Rendered:

\begin{tabular}{|c|c|}
\hline
\textbf{Name} & \textbf{Age} \\
\hline
Alice & 30 \\
Bob & 25 \\
Carol & 28 \\
\hline
\end{tabular}

Code:

\begin{verbatim}
\begin{tabular}{|c|c|}
\hline
\textbf{Name} & \textbf{Age} \\
\hline
Alice & 30 \\
Bob & 25 \\
Carol & 28 \\
\hline
\end{tabular}
\end{verbatim}

\section{References}

For further learning:

\begin{itemize}
    \item Overleaf Documentation: \texttt{https://www.overleaf.com/learn}
    \item LaTeX Wikibooks: \texttt{https://en.wikibooks.org/wiki/LaTeX}
    \item CTAN (Comprehensive TeX Archive Network): \texttt{https://ctan.org}
\end{itemize}

\end{document}
