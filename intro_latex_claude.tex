\documentclass[11pt]{article}
\usepackage{amsmath}  % For mathematical expressions
\usepackage{array}    % For better table formatting
\usepackage{color}    % For colored text
\usepackage{hyperref} % For hyperlinks
\usepackage{verbatim} % For displaying code

% Set up hyperref to have blue links
\hypersetup{
    colorlinks=true,
    linkcolor=blue,
    urlcolor=blue
}

\title{A Beginner's Guide to \LaTeX}
\author{Your Name Here}
\date{\today}

\begin{document}

\maketitle

\tableofcontents
\newpage

\section{Introduction}

Welcome to \LaTeX! This document will introduce you to the essential elements of \LaTeX{} typesetting. \LaTeX{} is a powerful document preparation system that helps you create professional-looking documents with minimal effort.

\subsection{What is \LaTeX?}

\LaTeX{} is a markup language that focuses on the content of your document rather than its appearance. You write your document in plain text and add commands that describe the structure. \LaTeX{} then formats your document automatically.

\section{Document Structure}

Every \LaTeX{} document starts with a document class declaration and is divided into two main parts:

\begin{enumerate}
    \item The \textbf{preamble}: Everything before \verb|\begin{document}|
    \item The \textbf{document body}: Everything between \verb|\begin{document}| and \verb|\end{document}|
\end{enumerate}

\subsection{Basic Document Structure}

Here's the minimal structure of a \LaTeX{} document:

\begin{verbatim}
\documentclass{article}
\begin{document}
Your content goes here.
\end{document}
\end{verbatim}

\section{Text Formatting}

\LaTeX{} provides various commands for formatting text:

\subsection{Basic Text Formatting}

\subsubsection{Bold and Italic Text}

To make text \textbf{bold}, use: \verb|\textbf{bold}|

To make text \textit{italic}, use: \verb|\textit{italic}|

To make text \textbf{\textit{bold and italic}}, use: \verb|\textbf{\textit{bold and italic}}|

\subsubsection{Text Sizes}

Different text sizes can be achieved using:

{\tiny This is tiny text} --- \verb|{\tiny This is tiny text}|

{\small This is small text} --- \verb|{\small This is small text}|

{\large This is large text} --- \verb|{\large This is large text}|

{\huge This is huge text} --- \verb|{\huge This is huge text}|

\subsubsection{Colored Text}

To use colored text, include the \verb|color| package and use:

{\color{red}This text is red} --- \verb|{\color{red}This text is red}|

{\color{blue}This text is blue} --- \verb|{\color{blue}This text is blue}|

\section{Lists}

\LaTeX{} supports several types of lists:

\subsection{Itemized Lists (Bullet Points)}

Here's an example of an itemized list:

\begin{itemize}
    \item First item
    \item Second item
    \item Third item with sub-items:
    \begin{itemize}
        \item Sub-item 1
        \item Sub-item 2
    \end{itemize}
\end{itemize}

The code for this list is:
\begin{verbatim}
\begin{itemize}
    \item First item
    \item Second item
    \item Third item with sub-items:
    \begin{itemize}
        \item Sub-item 1
        \item Sub-item 2
    \end{itemize}
\end{itemize}
\end{verbatim}

\subsection{Enumerated Lists (Numbered)}

Here's an example of an enumerated list:

\begin{enumerate}
    \item First step
    \item Second step
    \item Third step with sub-steps:
    \begin{enumerate}
        \item Sub-step 1
        \item Sub-step 2
    \end{enumerate}
\end{enumerate}

The code for this list is:
\begin{verbatim}
\begin{enumerate}
    \item First step
    \item Second step
    \item Third step with sub-steps:
    \begin{enumerate}
        \item Sub-step 1
        \item Sub-step 2
    \end{enumerate}
\end{enumerate}
\end{verbatim}

\section{Mathematical Expressions}

\LaTeX{} excels at typesetting mathematical expressions. You can include math in two ways:

\subsection{Inline Mathematics}

For math within text, use dollar signs: $E = mc^2$

Code: \verb|$E = mc^2$|

\subsection{Display Mathematics}

For centered, standalone equations, use:

\[
    \int_{0}^{\infty} e^{-x^2} dx = \frac{\sqrt{\pi}}{2}
\]

Code: 
\begin{verbatim}
\[
    \int_{0}^{\infty} e^{-x^2} dx = \frac{\sqrt{\pi}}{2}
\]
\end{verbatim}

\subsection{Common Mathematical Symbols}

Here are some common mathematical expressions:

\begin{itemize}
    \item Fractions: $\frac{a}{b}$ --- \verb|$\frac{a}{b}$|
    \item Subscripts: $x_i$ --- \verb|$x_i$|
    \item Superscripts: $x^2$ --- \verb|$x^2$|
    \item Square root: $\sqrt{x}$ --- \verb|$\sqrt{x}$|
    \item Sum: $\sum_{i=1}^{n} x_i$ --- \verb|$\sum_{i=1}^{n} x_i$|
    \item Greek letters: $\alpha, \beta, \gamma$ --- \verb|$\alpha, \beta, \gamma$|
\end{itemize}

\subsection{Numbered Equations}

For numbered equations, use the \verb|equation| environment:

\begin{equation}
    f(x) = ax^2 + bx + c
\end{equation}

Code:
\begin{verbatim}
\begin{equation}
    f(x) = ax^2 + bx + c
\end{equation}
\end{verbatim}

\section{Tables}

Tables in \LaTeX{} are created using the \verb|tabular| environment:

\subsection{Simple Table}

Here's a basic table:

\begin{table}[h]
\centering
\begin{tabular}{|l|c|r|}
\hline
\textbf{Left} & \textbf{Center} & \textbf{Right} \\
\hline
Item 1 & A & 100 \\
Item 2 & B & 200 \\
Item 3 & C & 300 \\
\hline
\end{tabular}
\caption{A simple table example}
\end{table}

The code for this table:
\begin{verbatim}
\begin{table}[h]
\centering
\begin{tabular}{|l|c|r|}
\hline
\textbf{Left} & \textbf{Center} & \textbf{Right} \\
\hline
Item 1 & A & 100 \\
Item 2 & B & 200 \\
Item 3 & C & 300 \\
\hline
\end{tabular}
\caption{A simple table example}
\end{table}
\end{verbatim}

\subsection{Table Column Specifications}

The column specification \texttt{\{|l|c|r|\}} part specifies:
\begin{itemize}
    \item \texttt{l} = left-aligned column
    \item \texttt{c} = center-aligned column
    \item \texttt{r} = right-aligned column
    \item \texttt{|} = vertical lines between columns
\end{itemize}

\section{Sections and Document Organization}

\LaTeX{} provides hierarchical sectioning commands:

\begin{verbatim}
\section{Section Title}
\subsection{Subsection Title}
\subsubsection{Subsubsection Title}
\paragraph{Paragraph Title}
\subparagraph{Subparagraph Title}
\end{verbatim}

You can also create unnumbered sections by adding an asterisk:
\begin{verbatim}
\section*{Unnumbered Section}
\end{verbatim}

\section{Special Characters}

Some characters have special meanings in \LaTeX{} and need to be escaped:

\begin{itemize}
    \item \% (percent) --- \verb|\%|
    \item \$ (dollar) --- \verb|\$|
    \item \& (ampersand) --- \verb|\&|
    \item \# (hash) --- \verb|\#|
    \item \_ (underscore) --- \verb|\_|
    \item \{ \} (braces) --- \verb|\{ \}|
    \item \textbackslash{} (backslash) --- \verb|\textbackslash{}|
\end{itemize}

\section{Comments}

You can add comments to your \LaTeX{} source that won't appear in the output:

\begin{verbatim}
This text will appear in the document.
% This is a comment and won't appear
\end{verbatim}

\section{References}

For further reading and learning, here are some excellent resources:

\begin{itemize}
    \item \href{https://www.latex-project.org/}{The LaTeX Project} --- Official LaTeX website
    \item \href{https://en.wikibooks.org/wiki/LaTeX}{LaTeX Wikibook} --- Comprehensive free guide
    \item \href{https://www.overleaf.com/learn}{Overleaf Documentation} --- Excellent tutorials and reference
    \item \href{https://tex.stackexchange.com/}{TeX Stack Exchange} --- Q\&A community for LaTeX
    \item \href{https://ctan.org/}{CTAN} --- Comprehensive TeX Archive Network
\end{itemize}

\section*{Conclusion}

This guide has introduced you to the basic elements of \LaTeX. With these foundations, you can create professional-looking documents. Remember:

\begin{itemize}
    \item \LaTeX{} separates content from formatting
    \item Use commands to structure your document
    \item Practice makes perfect!
\end{itemize}

Happy typesetting!

\end{document}